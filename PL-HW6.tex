\documentclass{article}

\usepackage{tikz} 
\usetikzlibrary{automata, positioning, arrows} 

\usepackage{amsthm}
\usepackage{amsfonts}
\usepackage{amsmath}
\usepackage{amssymb}
\usepackage{fullpage}
\usepackage{color}
\usepackage{parskip}
\usepackage{hyperref}
  \hypersetup{
    colorlinks = true,
    urlcolor = blue,       % color of external links using \href
    linkcolor= blue,       % color of internal links 
    citecolor= blue,       % color of links to bibliography
    filecolor= blue,        % color of file links
    }
    
\usepackage{listings}
\usepackage[utf8]{inputenc}                                                    
\usepackage[T1]{fontenc}                                                       

\definecolor{dkgreen}{rgb}{0,0.6,0}
\definecolor{gray}{rgb}{0.5,0.5,0.5}
\definecolor{mauve}{rgb}{0.58,0,0.82}

\lstset{frame=tb,
  language=haskell,
  aboveskip=3mm,
  belowskip=3mm,
  showstringspaces=false,
  columns=flexible,
  basicstyle={\small\ttfamily},
  numbers=none,
  numberstyle=\tiny\color{gray},
  keywordstyle=\color{blue},
  commentstyle=\color{dkgreen},
  stringstyle=\color{mauve},
  breaklines=true,
  breakatwhitespace=true,
  tabsize=3
}

\newtheoremstyle{theorem}
  {\topsep}   % ABOVESPACE
  {\topsep}   % BELOWSPACE
  {\itshape\/}  % BODYFONT
  {0pt}       % INDENT (empty value is the same as 0pt)
  {\bfseries} % HEADFONT
  {.}         % HEADPUNCT
  {5pt plus 1pt minus 1pt} % HEADSPACE
  {}          % CUSTOM-HEAD-SPEC
\theoremstyle{theorem} 
   \newtheorem{theorem}{Theorem}[section]
   \newtheorem{corollary}[theorem]{Corollary}
   \newtheorem{lemma}[theorem]{Lemma}
   \newtheorem{proposition}[theorem]{Proposition}
\theoremstyle{definition}
   \newtheorem{definition}[theorem]{Definition}
   \newtheorem{example}[theorem]{Example}
\theoremstyle{remark}    
  \newtheorem{remark}[theorem]{Remark}

\title{CPSC-354 Report}
\author{Davis Wood  \\ Chapman University}

\date{\today} 

\begin{document}

\maketitle

\begin{abstract}
\end{abstract}

\setcounter{tocdepth}{3}
\tableofcontents

\section{Solution}\label{intro}
$let rec fact = \lambda n. if n=0 then 1 else n * fact (n-1) in fact 3$ \\
-> $let fact = (fix(\lambda fact.\lambda n.if n=0 then 1 else n * fact (n-1))) in fact 3$ \\
-> $(\lambda fact.\lambda n.fact 3) (fix(\lambda fact.if n=0 then 1 else n * fact (n-1)))$ \\
-> $fix(\lambda fact.\lambda n.if n=0 then 1 else n * fact (n-1)) 3$ \\
-> $((\lambda fact.\lambda n.if n=0 then 1 else n * fact (n-1)) (fix(\lambda fact.\lambda n.f n=0 then 1 else n * fact (n-1)))) 3$ \\
-> $(\lambda n.(if n=0 then 1 else n * (fix(\lambda fact.\lambda n.if n=0 then 1 else n * fact (n-1))) (n-1))) 3$ \\
-> $(if 3=0 then 1 else 3 * (fix(\lambda fact.\lambda n.if n=0 then 1 else n * fact (n-1))) (3-1))$ \\
-> $3 * (fix(\lambda fact.\lambda n.if n=0 then 1 else n * fact (n-1))) 2$ \\
-> $3 * ((\lambda fact.\lambda n.if n=0 then 1 else n * fact (n-1)) fix(\lambda fact.\lambda n.if n=0 then 1 else n * fact (n-1))) 2$ \\
-> $3 * ((\lambda n.if n=0 then 1 else n * (fix(\lambda fact.\lambda n.if n=0 then 1 else n * fact (n-1))) (n-1))) 2$ \\
-> $3 * ((if 2=0 then 1 else 2 * (fix(\lambda fact.\lambda n.if n=0 then 1 else n * fact (n-1))) (2-1)))$ \\
-> $3 * 2 * (fix(\lambda fact.\lambda n.if n=0 then 1 else n * fact (n-1))) (1)$ \\
-> $3 * 2 * ((\lambda fact.\lambda n.if n=0 then 1 else n * fact (n-1)) fix(\lambda fact.\lambda n.if n=0 then 1 else n * fact (n-1))) (1)$ \\
-> $3 * 2 * ((\lambda n.if n=0 then 1 else n * (fix(\lambda fact.\lambda n.if n=0 then 1 else n * fact (n-1))) (n-1))) (1)$ \\
-> $3 * 2 * ((if 1=0 then 1 else 1 * (fix(\lambda fact.\lambda n.if n=0 then 1 else n * fact (n-1))) (1-1)))$ \\
-> $3 * 2 * 1 * (fix(\lambda fact.\lambda n.if n=0 then 1 else n * fact (n-1))) (0)$ \\
-> $3 * 2 * 1 * ((\lambda fact.\lambda n.if n=0 then 1 else n * fact (n-1)) fix(\lambda fact.\lambda n.if n=0 then 1 else n * fact (n-1))) (0)$ \\
-> $3 * 2 * 1 * ((\lambda n.if n=0 then 1 else n * (fix(\lambda fact.\lambda n.if n=0 then 1 else n * fact (n-1))) (n-1))) (0)$ \\
-> $3 * 2 * 1 * ((if 0=0 then 1 else 0 * (fix(\lambda fact.\lambda n.if n=0 then 1 else n * fact (n-1))) (0-1))) (0)$ \\
-> $3 * 2 * 1 * 1$ \\
-> $6$
\end{document}
